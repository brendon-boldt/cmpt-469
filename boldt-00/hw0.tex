\documentclass[a4paper]{article}
\usepackage[letterpaper, margin=1in]{geometry} % page format
\usepackage{listings} % this package is for including code
\usepackage{graphicx} % this package is for including figures
\usepackage{amsmath}  % this package is for math and matrices
\usepackage{amsfonts} % this package is for math fonts
\usepackage{tikz} % for drawings
\usepackage{hyperref} % for urls
\usepackage{enumitem} % for fancy enumerates

\title{Deep Learning - Homework 0}
\author{Brendon Boldt}
\date{Sep/11/2017}

\begin{document}
\lstset{
    language=Python,
    frame=single,
    breaklines=true,
    postbreak=\mbox{\textcolor{gray}{$\hookrightarrow$}\space} 
}

\maketitle

\section{Course Preparation}

\subsection{Python Requirements}
The following source code demonstrates the installation of the required
Python libraries:

\lstinputlisting[language=Python,frame=single]{hw0.libraries.py}

\begin{lstlisting}
> python3 hw0.libraries.py
3.5.2 (default, Nov 17 2016, 17:05:23) 
[GCC 5.4.0 20160609]
1.13.1
0.19.1
0.19.0
2.0.2
0.20.3
1.3.0
\end{lstlisting}

\subsection{GitHub Repository}

My GitHub profile can be found at
\url{https://github.com/brendon-boldt}, 
and the class repository at
\url{https://github.com/brendon-boldt/cmpt-469}.
The following screenshot is not fabricated

\begin{center}
    \includegraphics[scale=0.5]{{hw0.github-invitation}.png}
\end{center}

\subsection{Kaggle Account}

My Kaggle username is \texttt{brendonboldt},
and the following screenshot is taken from my account page:

\begin{center}
    \includegraphics[scale=0.5]{{hw0.kaggle}.png}
\end{center}


\section{Solution to Problem 1}
\textit {
    For the function $g(x) = -3x^2 + 24x - 30$, find the value for $x$ that maximizes $g(x)$.
} \vspace{5mm}


We know that $g'(x) = -6x + 24$; we can then see that $g(x)$ has one critical point at $x = 4$. Since $g'(x) > 0$ for $x < 4$ and $g'(x) < 0$ for $x > 4$ we know that the critical point at $x = 4$ is a local maximum.
Since $g(x)$ is of degree $2$, it has only one local maximum or minimum; we also know that $\lim_{x \to -\infty} = -\infty$ and $\lim_{x \to \infty} = -\infty$. Thus, the local maximum at $x = 4$ is the global maximum.

\section{Solution to Problem 2}
\textit{
    Consider the following function:
    \[ f(x) = 3x^3_0 - 2x_0x^2_1 + 4x_1 - 8 \]
    what are the partial derivatives of $f(x)$ with respect to $x_0$ and $x_1$?
} \vspace{5mm}

It is the case that
$\frac{\partial f}{\partial x_0} = 9x_0^2 - 2x^2_1$ and
$\frac{\partial f}{\partial x_1} = -4x_0x_1 + 4$.


\section{Solution to Problem 3}
\textit {
    Consider the matrix $A = \begin{bmatrix}1&4&-3\\2&-1&3\end{bmatrix}$ and $B = \begin{bmatrix}-2&0&5\\0&-1&4\end{bmatrix}$, then answer the following and verify your answers in Python:
    \begin{enumerate}[label=(\alph*)]
        \item can you multiply the two matrices? Elaborate on your answer.
        \item multiply $A^T$ and $B$ and give its rank.
        \item let $C = \begin{bmatrix}1&0\\0&2\end{bmatrix}$ be a new matrix; what is the result of $AB^T + C^{-1}$
    \end{enumerate}
} \vspace{5mm}

\begin{enumerate}[label=(\alph*)]
    \item No; it is not possible to multiply a $2 \times 3$ and a $2 \times 3$ matrix.
    \item
        \begin{align*}
            A^TB &= \begin{bmatrix}1&2\\4&-1\\-3&3\end{bmatrix}
            \begin{bmatrix}-2&0&5\\0&-1&4\end{bmatrix} \\
            &= \begin{bmatrix}-2+0&0+-2&5+8\\-8+0&0+1&20+-4\\6+0&0+-3&-15+12\end{bmatrix} \\
            &= \begin{bmatrix}-2&-2&13\\-8&1&16\\6&-3&-3\end{bmatrix}
        \end{align*}
    \item 
        \begin{align*}
            AB^T + C^{-1} &=
            \begin{bmatrix}1&4&-3\\2&-1&3\end{bmatrix}
            \begin{bmatrix}-2&0\\0&-1\\5&4\end{bmatrix}
            + \frac{1}{2}\begin{bmatrix}2&0\\0&1\end{bmatrix} \\
            &= \begin{bmatrix}-17&-16\\11&13\end{bmatrix}
            + \begin{bmatrix}1&0\\0&\frac12\end{bmatrix} \\
            &= \begin{bmatrix}-16&-16\\11&\frac{27}2\end{bmatrix}
        \end{align*}
        
    The results can be confirmed by running the following code:
\lstinputlisting{hw0.math.py}
\begin{lstlisting}[]
> python3 hw0.math.py
numpy

a^T b =
[[ -2.  -2.  13.]
 [ -8.   1.  16.]
 [  6.  -3.  -3.]]

a b^T + c^-1 =
[[-16.  -16. ]
 [ 11.   13.5]]

tensorflow

a^T x b =
[[ -2.  -2.  13.]
 [ -8.   1.  16.]
 [  6.  -3.  -3.]]

a b^T + c^-1 =
[[-16.  -16. ]
 [ 11.   13.5]]
\end{lstlisting}
\end{enumerate}


\section{Solution to Problem 4}
\textit{
    Suppose that random variable $X ~ N(2,3)$. What is the expected value of $X$.
} \vspace{5mm}

The expected value of $X$ is $2$ since the normal distribution, which is symmetric, described by $X$ is centered about $2$.

\end{document}
