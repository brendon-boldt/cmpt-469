\documentclass[a4paper]{article}
\usepackage[letterpaper, margin=1in]{geometry} % page format
\usepackage{listings} % this package is for including code
\usepackage{graphicx} % this package is for including figures
\usepackage{amsmath}  % this package is for math and matrices
\usepackage{amsfonts} % this package is for math fonts
\usepackage{tikz} % for drawings
\usepackage{hyperref} % for urls
\usepackage{enumitem} % for fancy enumerates
\usepackage{cite}
\usepackage{syntax}
\usepackage{scrextend}

\title{Milestone: Augmenting Interactive Semantic Parsing with Word Embeddings}
\author{Brendon Boldt}
\date{Nov/13/2017}

\begin{document}
\lstset{
    language=Python,
    frame=single,
    breaklines=true,
    postbreak=\mbox{\textcolor{gray}{$\hookrightarrow$}\space} 
}

% You will activate the laser beams, and you and your knapsack will be shuredded into pieces


% http://www.aclweb.org/anthology/W14-2405
% Not sure if this is useful to me

% Look at https://nlp.stanford.edu/pubs/SocherBauerManningNg_ACL2013.pdf

\maketitle

%\cite{DBLP:journals/corr/abs-1301-3781}.

\section*{Abstract}

Semantic parsing provides a powerful way to demonstrate understanding of a
natural language since it constructs easily executable formal
representations from a natural language utterance. One issue with semantic parsing, though,
is that it treats words merely as tokens and has no mechanism for inferring
the semantics of an unknown word. Using vector representations of words
in semantic parsing provides an effective solution to this problem since
it allows the semantic parser to infer the meaning of new words by comparing
their similarity to known words. Furthermore, vector word embeddings can
be learned at a large scale since it is an unsupervised algorithm.

\section{Introduction}

One of the major subfields of natural language processing is natural language
\textit{understanding}, which carries the connotation that the system processing
the language is able to find meaning in an utterance rather than simply
doing something such as searching for keyword occurences or otherwise. % Cite?
One such method of demonstrating an understanding of a natural language is
by having the system in question produce a concrete action or piece of
information which has semantic matching that of the utterance.

Semantic parsing, on a general level, is an effective way of demonstrating an
understanding of natural language because it maps utterances to
formal representations (e.g., lambda-DCS, SQL) based on a formal grammar.
Semantic parsing has been applied to tasks such as answering questions,
interpreting email-related commands, and building virtual block structures.
\cite{AAAI1612383}
\cite{DBLP:conf/emnlp/BerantCFL13}
\cite{DBLP:journals/corr/WangGLM17}.
% In such applications, it useful to map semantic
%%%%%%%%%%%%%%%%%%%%%%%%%%%%%%%%%%%%%%%%%%%%%%%%%%%%%%%%%%%%%%%%%%%%%%%%%%%%%%%%

Like most formal grammars, semantic parsing grammars usually treat each word
as either a unique or lemmatized\footnote{Lemmatized tokens account for word
inflections; for example, ``walks'', ``walked'', and ``walking'' would all
match ``walk'' when lemmatized.} token.
In this model, each word has no meaning until it is explicitly assigned one
by the semantics of the grammar.

This inflexibility is important in the contexts of applications like programming
languages where exactness is expected and required, but this trait is not so
helpful when parsing natural language.
For example, a semantic parser might interpret ``select red box'' while not
understanding ``choose red box'' which, for the large part, mean the same thing.
Understanding natural language requires flexibility in that words with
similar meaning should have \textit{some} degree of interchangeability as
opposed to none which is offered by the typical token-based approach.

%%%%%%%%%%%%%%%%%%%%%%%%%%%%%%%%%%%%%%%%%%%%%%%%%%%%%%%%%%%%%%%%%%%%%%%%%%%%%%%%
It would be better, then, to represent words in a way that would group words
with similar meanings, and word embeddings, that is, representing words as
vectors gives us a very efficient way to do this.
One of the primary advantages of augmenting semantic parsing with word
embeddings as opposed to dictionary synonyms is that learning the embeddings
is a unsupervised task which allows it to be done at scale much more easily than
hand-picking sets of words which are semantically similar enough to a given
token.

\section{Methodology}

%~%~% Where do I describe Voxelurn?
For the purposes of this paper, we ill be using the following grammar
which is a simplified and slightly altered version of the
of the grammar used in Voxelurn \cite{DBLP:journals/corr/WangGLM17}.

\vspace{5mm}
\begin{addmargin}[1cm]{0cm}
    \begin{grammar}

    <Root> ::= <Action>

    <Action> ::= <ActionVerb> <Object>
    
    <ActionVerb> ::= `add' | `remove' | `select'

    <Object> ::= <ObjectItem> | <Color> <ObjectItem>

    <ObjectItem> ::= `ball' | `box'
    
    <Color> ::= `light' | `dark'
    
    \end{grammar}
\end{addmargin}
\vspace{5mm}

Using a simple grammar such as the one above allows us to isolate a few grammar
rules to test the feasibility of using word embeddings to augment semantic
parsing.

Normally, parsing uses a standard bottom-up approach where the terminal tokens
are matched with the corresponding tokens in the utterance; if one of the
tokens in the utterance does not match, the parse will fail.
In the augmented semantic parser, an unknown token is treated as a wildcard
so that the parse succeeds. The unknown token is then compared with the
tokens in the grammar using the word embeddings to determine which grammatical
token to use.

We are using the GloVe word vector model as a source for word embeddings
\cite{pennington2014glove}.
To measure the degree of similarity between two words, we are using the consine
similarity of two corresponding vectors
[source to be added].
%~%~% Cosine similarity paper

%%%%%%%%%%%%%%%%%%%%%%%%%%%%%%%%%%%%%%%%%%%%%%%%%%%%%%%%%%%%%%%%%%%%%%%%%%%%%%%%



%~%~% Paper with similarity metric?
% https://nlp.stanford.edu/~manning/xyzzy/acl30224.pdf

\section{Experiment}

The experiement will consist of taking valid sentences in the grammar and
replacing individual words with synonyms and determing whether the augmented
semantic parser correctly infers the correct parse given the unknow token.
For example, take the utterance ``build box''; here we have an $\langle ActionVerb\rangle$
replaced with the word ``build''.
The parser would make a correct inference if it maps ``build'' to the correct
grammatical $\langle ActionVerb\rangle$ which would be ``add'' in this case (and not
``remove'' or ``select'') since ``build'' and ``add'' are loose synonyms.
For this experiement, WordNet will be used to generate the synonyms.

Further experimentation would involve using human subjects (e.g., using
Amazon Mechanical Turk) to directly test whether the robustness added by the
word embeddings is effective or not.
The effectiveness would be determined by the ability of the semantic parser
to address the natural variations in language as used by humans rather than
just the strict synonym replacement demonstrated in this paper.

\section{Discussion}

\section{Conclusion}


\section{Progress}

I am currently in the process of modifying the SEMPRE semantic parser to 
parse an unknown tokens as a wildcard so that the candidate tokens can be
compared against the unkown token using the word embedding.
I have currently written the code that does the word embedding comparison based
on the GloVe pre-trained word embeddings.
I plan to use WordNet to generate the majority of synonyms that will be used to
test the word embedding semantic parser;
specifically, the Java WordNet interface provided here:
\url{https://rednoise.org/rita/reference/RiWordNet.php}
will be used.


\iffalse
\section{Proposal}

For my semester project, I plan to take the Voxelurn natural language system
and making more robust using vector representation of words
Voxelurn, at its core, starts with a ``core language'' (which is itself a formal
language) which is able to describe basic block-building actions
(e.g., add, remove, select) and allows users to extend the language as they use
it. This extension consists of allowing users to define new commands in terms
of already known commands; the newly defined commands are added to the grammar
of the formal language. The purpose of this extension is that language becomes
more natural as people use the system.

The words in formal language are simply represented as individual tokens.
In this way, a command such as ``make red block'' might be recognized while
something like ``build red block'' would not be recognized at all.
Using vector representation of words
would allow the language system to be more robust since words with
sufficiently similar vectors could be used in place of each other. Thus,
words like ``make'' and ``build'' could be used in place of each other
without each command needing to be explicitly learned.
Normally the right-hand-side of formal language production rules consists
of strings of symbolic tokens; this approach, instead, would turn the
right-had-side of production rules into sequences of vectors in order
to improve the flexibility of the language as a whole.

\section{Tools}

A demo and explanation of Voxelurn (as well as a link to the paper)
can be found at \url{http://www.voxelurn.com/#/about}. The formal
language and parsing aspect of the project is based on the SEMPRE
project from Stanford University. Much of the
Voxelurn project will be left as-is; though the parser will have
to be modified in order to support vector representations of words.
The word vectorization will be based on the TensorFlow machine
learning library and be derived from the tutorial found at 
\url{https://www.tensorflow.org/tutorials/word2vec}.
\section{Datasets}

While I have not completely decided on particular datasets, I hope
to largely use datasets that do not require extensive preprocessing.
For the word vectorization, it would be ideal if a standard English corpus (such as the Penn Treebank) could provide the necessary
vectorization of words. The Voxelurn project also contains sample
datasets of user-entered commands; these could possibly be augmented
by replacing words with dictionary synonyms in order to test the
word vectorization's effectiveness.
\fi

\bibliography{bibliography}{}
\bibliographystyle{plain}

\end{document}
